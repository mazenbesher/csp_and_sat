% !TeX spellcheck = en_US
\section{SAT into CSP encodings}\label{sec:sat_to_csp}
In this section, we discuss the opposite direction by introducing several ways for encoding SAT into CSP and briefly summarize the pros and cons of each encoding.

\subsection{Dual encoding}
For each clause in the CNF-formatted SAT a CSP variable is generated. The domain of each variable consists of the possible truth assignments that satisfies the corresponding clause expressed as a tuple. Any two clauses sharing the same propositional variable must also have a binary constraint between their respective dual variables. These constraints assure consistent assignment of the common propositional variable such that no variable is assigned two different values at the same time. 

The following example shows how we could encode a SAT problem with two clauses and one common propositional variable. Considering the clauses: $x_1 \vee x_2$ and $x_3 \vee \neg x_2$. The first clause results in the dual variable $D_1$ with the domain $\{(T,T),(T,F),(F,T)\}$ where $T$ and $F$ represent the truth value of the corresponding Boolean variable at the respective position within the tuple. The second clause results in $D_2 \in \{(T,F),(T,T),(F,F)\}$. Since $x_2$ appears in both clauses, we must ensure that it will not have different values in each variable through the binary constraint: $D_1[2] = \neg D_2[2]$ which assures that the second element of the tuple assigned to $D_1$ is the inverse of the second element of the $D_2$ tuple.

\subsubsection{Theoretical comparison}
It has been proven \cite{walsh2000sat} that FC on the dual encoded SAT instance does more work than DPLL on the original problem. The proof is based on the fact that a truth assignment under unit propagation in DPLL implies an elimination of contradictory values on the CSP side. By induction, we can see that generating the empty clause in DPLL (which indicates that the problem is unsatisfiable regarding the current branch) implies a domain wipe out after applying arc-consistency by FC.

\subsection{Hidden variable encoding}
Just like dual encoding, the hidden variable encdoing assign a dual variable for each SAT clause. Furthermore, each propositional variable $x_i$ is simulated on the CSP side with a binary variable $X_i \in \{T,F\}$. This encodings allows constraints to be easily formulated in terms of thier binary variables. For example the clause $x_1 \vee \neg x_2$ results in one dual variable $D_1 \in \{(T,F),(F,F),(T,T)\}$ and two binary CSP variables $X_1, X_2$ and additionally the coupling constraints $D_1[1] = x_1$ and $D_1[2] = \neg x_2$ which constraints any assignment of the elements of the tuple assigned to $D_1$ to be consistent with the propositional variables $x_1,x_2$ represented by their CSP counterparts $X_1,X_2$.

\subsubsection{Theoretical comparison}
The hidden variable encoding maintain equivalent relation between assigning truth values committed by one literal rule on the DPLL side and eliminating contradictory values by enforcing arc-consistency by FC/MAC on the node level. This translates to equivalence relation between generating empty clauses and empty domain \cite{walsh2000sat}. This in turn means that DPLL will essentially branch in general as much as FC/MAC applied to the encoded version of the same problem, and the same amount of work is expected by each algorithm.

\subsection{Literal encoding}
The literal encoding express each clause as a variable with a domain consisting of propositional variables' values that satisfy the corresponding clause. Constraints are required between any CSP variables if and only if their domains contain contradictory propositional variables. For example the clauses $x_1 \vee x_2$ and $x_3 \vee \neg x_2$ can be expressed as the following variables $D_1 \in \{x_1, x_2\}, D_2 \in \{x_3, \neg x_2\}$ and the constraint $\neg (D_1 = x_2 \wedge D_2 = \neg x_2)$ which rules out this inconsistent assignment for $x_2$.

\subsubsection{Theoretical comparison}
This encoding shows similar results as the hidden variable encoding regarding the relation between generating the empty clause and empty domains as a result of enforcing arc-consistency. Anyway MAC is strictly dominated by DPLL applied to the original. Strictness can be proved easily by contradiction. Consider any unsatisfiable $k-$SAT instance (i.e. a SAT problem with $k$ propositional variable) with $2^k$ clauses where $k > 2$. DPLL needs $2^{k-1}$ branch to arrive at the empty clause while MAC takes $k!$ branches regardless of the used heuristics \cite{walsh2000sat}.

\subsubsection{Advantage of literal encoding}
CSP domain size can have great impact on the performance of any arc-consistency based algorithm (like MAC or FC) and must be taken into consideration. Let $n$ be the number of SAT clauses. Dual and hidden variable encodings generate CSP instance with domain size in the order of $\mathcal{O}(2^n)$ while the literal encoding generate CSP instances with domain size of $\mathcal{O}(n)$
 