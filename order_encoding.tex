% !TeX spellcheck = en_US
\subsection{Order encoding}

Order encoding took a different path to approach the problem of transforming CSP into SAT. The essential idea is extending the expressive power of Boolean expressions to represent a whole range of possible values on the CSP side as a comparison.

Despite being simple in its complexity, the order encoding is one if most efficient approaches to encode CSP into SAT and generate a relatively simpler instance that can be solved quickly under the assumption that the right conditions are met (see \ref{subsec:order_encoding_imprtance}).

\subsubsection{Usage}
The core mechanism of many constraint solvers is based on another SAT-Solver (such as MiniSat\cite{tamura2008sugar}). The given CSP is first encoded using order encoding (or a variant of it) and then solved by the SAT-Solver. One of the first prominent award-winning SAT-based constraint solvers called sugar. It is based around the same idea using MiniSat as the underlying SAT-solver \cite{tamura2008sugar}.

\subsubsection{Encoding the variables}
Assuming a consistent CSP domain for the variable $x$ ranging from $l(x)$ to $u(x)$ where $l$ and $u$ represent the lower and upper bounds respectively, a Boolean variable is generated $p_{xi}$ for each value $i$ in the domain of $x$. $p_{xi}$ is semantically equal to $x \leq i$, and therefor $p_{xk}$ for $k = u(x)$ is always True and must be added as its own clause because it must be satisfied at all times.

To facilitate an unambiguous entry point for DPLL, the order encoding adds additional trivial propositional variable  $p_{xj}$ for each CSP  variable $x$ where $j = l(x) - 1$. This variable must be added as its own clause since it's clearly False at all times. Similar variables of the form $p_{xk}$ for  $k = u(x)$ are always True. Both of these clauses $\neg p_{xj} \wedge p_{xk}$ helps the unit propagation rule exploiting the SAT instance of the problem and constraint it by providing boundaries on each variable.

Since $x \leq i$ implies $x \leq i+1$ and $p_{xi}$ means $x \leq i$, we must also encode this fact into our SAT to ensure a consistent assignment of each variable:
$$
x \leq i \rightarrow x \leq i + 1 \quad \equiv \quad
\neg x \leq i \vee x \leq i + 1 \quad \equiv \quad
\neg p_{xi} \vee p_{x(i+1)}
$$

\begin{figure}[H]
	\centering
	\includegraphics[width=0.7\linewidth]{assets/order_encoding_variables}
	\captionsetup{justification=centering,margin=2cm}
	\caption{An example for encoding CSP variables using order encoding}
	\label{fig:order_encoding_variables}
\end{figure}

\subsubsection{Encoding inequality constraints}
Order encoding works best for the inequality constraints (concretely constraints of the form $\sum_i a_i x_i \leq b$). Other types of constraints will be discussed in the next section \ref{sec:general_constraints}. The efficiency of the order encoding lays in its powerful encoding of entire conflict regions instead of single conflict points (in comparison with log and direct encodings) \cite{tamura2008sugar}. To encode a constraint of the form $\sum_{i=1}^n a_i x_i \leq b$, we must first generate the conflict regions for each variable which is a set of integer tuples $C := \{c_1, \dots, c_n \}$ that satisfies the constraints: 
\begin{align}
	&\sum_{i=1}^n c_i = b - n + 1 \\
	&\bigwedge_{i=1}^n (\min(a_i \cdot x_i) \leq c_i \leq \max(a_i \cdot x_i))
\end{align}

Then the required clauses to encode the constraint are:
\begin{equation}
\forall c_i \in C: \bigvee_{i=1}^n 
\begin{dcases}
	x_i \leq \lfloor \frac{c_i}{a_i} \rfloor \equiv
	 p_{x_i (\lfloor \frac{c_i}{a_i} \rfloor)} &\quad a_i > 0 \\[10pt]
	\neg (x_i \leq \lceil \frac{c_i}{a_i} - 1) \rceil \equiv
	 \neg p_{x_i (\lceil \frac{c_i}{a_i} - 1) \rceil)} &\quad a_i < 0 
\end{dcases}
\end{equation}

\paragraph*{Example}
To encode the constraint $x + y \leq 5$ for the unified domain $\{2,3,4\}$ we need to consider the conflict regions $C = \{ (1,3), (2,2), (3,1) \}$, so the following clauses must be included in our SAT formula:
$$
	p_{x1} \vee p_{y3}  \quad(\bullet) \qquad
	p_{x2} \vee p_{y2}  \quad(\times)  \qquad
	p_{x3} \vee p_{y1}  \quad(\circ)
$$

\begin{figure}[H]
	\centering
	\definecolor{cqcqcq}{rgb}{0.7529411764705882,0.7529411764705882,0.7529411764705882}
	\begin{tikzpicture}[line cap=round,line join=round,>=triangle 45,x=1.0cm,y=1.0cm]
	\draw [color=cqcqcq,, xstep=1.0cm,ystep=1.0cm] (0,0) grid (5.5,5.5);
	\draw[->,color=black] (-0.78,0.) -- (5.75,0.);
	\foreach \x in {,1.,2.,3.,4.,5.}
	\draw[shift={(\x,0)},color=black] (0pt,2pt) -- (0pt,-2pt) node[below] {\footnotesize $\x$};
	\draw[->,color=black] (0.,-0.57) -- (0.,5.92);
	\foreach \y in {,1.,2.,3.,4.,5.}
	\draw[shift={(0,\y)},color=black] (2pt,0pt) -- (-2pt,0pt) node[left] {\footnotesize $\y$};
	\draw[color=black] (0pt,-10pt) node[right] {\footnotesize $0$};
	\clip(-0.78,-0.57) rectangle (5.75,5.91);
	\draw [domain=-0.78:5.75] plot(\x,{(--5.-1.*\x)/1.});
	\begin{scriptsize}
	\draw [fill=black] (1.,3.) circle (2.5pt);
	\draw [fill=black] (1.,2.) circle (2.5pt);
	\draw [fill=black] (1.,1.) circle (2.5pt);
	\draw [color=black] (2.,2.)-- ++(-2.5pt,-2.5pt) -- ++(5.0pt,5.0pt) ++(-5.0pt,0) -- ++(5.0pt,-5.0pt);
	\draw [color=black] (2.,1.)-- ++(-2.5pt,-2.5pt) -- ++(5.0pt,5.0pt) ++(-5.0pt,0) -- ++(5.0pt,-5.0pt);
	\draw [color=black] (3.,1.) circle (2.5pt);
	\end{scriptsize}
	\end{tikzpicture}
	\captionsetup{justification=centering,margin=2cm}
	\caption{An example for encoding the constraint $x + y \leq 5$ using order encoding}
	\label{fig:order_encoding_example}
\end{figure}

\subsubsection{Encoding general constraints}\label{sec:general_constraints}
Other types of constraints are replaced by equivalent expressions composed from one or multiple constraints of the form $\sum_i a_i x_i \leq b$ while adding extra conditions (clauses) if necessary.

For example to encode the expression $x < y$ we replace it by $x + 1 \leq y \equiv p_{(x+1)y}$, an equality expression between $x$ and $y$ can be then replaced by $x \leq y \wedge x \geq y \equiv x \leq y \wedge \neg (x < y) \equiv p_{xy} \wedge p_{(x+1)y}$, similarly $x \neq y \equiv x < y \vee x > y$ and so on.

The following table shows how the CSP-solver sugar tries to replace general expressions with constraints of the form $\sum_i a_i x_i \leq b$ while adding extra conditions if needed \cite{tamura2008sugar}.

\begin{center}
\begin{tabular}{|c|c|c|}
	\hline 
	\textbf{Original Constraint} & \textbf{Replacement} & \textbf{Extra condition} \\ 
	\hline 
	$\max(a,b)$ & $x$ & $(x \geq a) \wedge (x \geq b) \wedge ((x \leq a) \vee (x \leq b))$ \\ 
	\hline 
	$\min(a,b)$ & $x$ & $(x \leq a) \wedge (x \leq b) \wedge ((x \geq a) \vee (x \geq b))$ \\ 
	\hline 
	$\left| a \right|$ & $\max(a,-a)$ &  \\ 
	\hline 
	$a\mod b$ & $r$ & $(a = c q + r) \wedge (0 \leq r) \wedge (r < c)$ \\ 
	\hline 
\end{tabular} 
\end{center}

\subsubsection{Importance}\label{subsec:order_encoding_imprtance}
In theory and assuming worst-time performance, the order encoding is no better than the other type of encodings, but in real-world applications it has been shown that the order encoding have better performance than the direct encoding on instances of the graph coloring and open-shop scheduling problems. \cite{tamura2009compiling}

Moreover, the order encoding, unlike the other encodings discussed earlier, is the first encoding that can translate a CSP into a so called tractable SAT, which means that the SAT problem can be solved by a SAT-solver in polynomial time. However, this statement is true only for certain classes of CSP problems. It has been proven that the following tractable CSP classes can be encoded into a tractable SAT problems using the order encoding: \cite{petke2011order}
\begin{enumerate}
	\item constraint-closed
	\item max-closed
	\item connected row-convex
\end{enumerate}
All the previous classes are some type of formality imposed on the constraints' set of the CSP problem. Comparisons and tables of SAT-solvers performance on the direct, log and order encoding of different tractable CSP instances can be found in \cite{petke2011order}.


